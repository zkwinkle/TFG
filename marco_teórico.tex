\chapter{Marco de referencia teórico}

Este capítulo introduce los conceptos teóricos necesarios para respaldar este
proyecto, el cual se sitúa en la intersección entre las áreas de ALS y ML. La
integración de estas disciplinas permite explorar nuevos métodos para la
generación de circuitos digitales que intercambien exactitud por eficiencia y
complejidad.

Se introducen conceptos fundamentales de ML, necesarios para entender los
mecanismos mediante los cuales un modelo puede ser útil en la generación de
circuitos aproximados, siendo especialmente relevante su capacidad para
aprender y generalizar funciones booleanas. Por otro lado, se revisan los
conceptos básicos de la computación aproximada y las técnicas tradicionales de
ALS, con el objetivo de contrastarlas con los enfoques basados en ML.

Finalmente, se abordan técnicas para transformar modelos de ML que han
aprendido funciones booleanas en circuitos lógicos, paso necesario para su
aplicación práctica dentro del marco de ALS.

La Figura \ref{fig:mapa_conceptual} mapea los conceptos tratados en
este capítulo en un mapa conceptual.

\begin{figure}[ht]
  \centering
  \includesvg[width=0.85\linewidth]{./imágenes/Mapa conceptual.svg}
  \caption{Mapa conceptual de los temas abordados en el marco teórico. Las burbujas rojas corresponden a temas asociados a ML, las burbujas azules corresponden a temas asociados a ALS y las burbujas moradas corresponden a temas que integran las 2 áreas.}
  \label{fig:mapa_conceptual}
\end{figure}

\section{Aprendizaje Automático}

Definición y objetivos.

\subsection{Aprendizaje supervisado}

\subsection{Aprendizaje reforzado}

\subsection{Generalización}

\subsection{Sobreajuste}

\subsection{Validación de modelos}

\subsection{Modelos y técnicas}

\subsubsection{Bosque aleatorio (RF)}

\subsubsection{Redes de tablas de búsqueda (LUTs)}

\subsubsection{Perceptrón multicapa / Red neuronal}

\subsubsection{Programación genética cartesiana (CGP)}


\section{Computación Aproximada}

\subsection{Síntesis Lógica Aproximada}

Definición y características

\subsection{Métodos clásicos}

Principios teóricos

\subsection{Métricas de error}

\subsection{Síntesis de modelos de ML}

\subsubsection{Diagramas de decisión binarios}

\subsubsection{Bosques de decisión}

\subsubsection{Redes neuronales}
