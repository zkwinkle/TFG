\chapter{Introducción}

Este proyecto consiste en la aplicación de técnicas de aprendizaje automático
(ML por sus siglas en inglés) al campo de la Síntesis Lógica Aproximada (ALS
por sus siglas en inglés).
Este informe detalla el proceso para la integración de una técnica del estado del
arte de ML en la herramienta de fuente abierta AxLS.

Este capítulo presenta el contexto y la motivación detrás del desarrollo del
proyecto. Se describen los antecedentes relevantes, el problema identificado,
así como los objetivos y alcances planteados para dar respuesta a dicho
problema.

\section{Antecedentes del proyecto}

En esta sección se recopila información relevante sobre la organización
involucrada, el área temática del proyecto y trabajos relacionados que han
abordado problemáticas similares. Esto permite ubicar el proyecto dentro de un
marco de referencia académico.

\subsection{Descripción de la organización}

Se realizará en colaboración con la escuela de Ingeniería Electrónica del
Instituto Tecnológico de Costa Rica, específicamente con el laboratorio de
"Efficient Computing Across the Stack" (ECASLab).

El Tecnológico de Costa Rica es una de las universidades públicas de Costa
Rica, con su sede original en Cartago creada en 1971 \cite{resena_tec}.

\subsection{Descripción del área de conocimiento del proyecto}

Este proyecto trata temas en la intersección de las áreas de computación
aproximada y aprendizaje automático (ML por sus siglas en inglés).
Específicamente en el campo generación de circuitos aproximados, actividad a la
que se le llama Síntesis de Lógica Aproximada (ALS por sus siglas en inglés).
Este proyecto plantea aplicar técnicas del área de aprendizaje automático al
área de ALS.

\subsection{Trabajos similares encontrados}

En \cite{castro-godinez_axls_2021}, Castro-Godínez et al. presentan una
herramienta de fuente abierta para ALS llamada AxLS. Este framework implementa
múltiples técnicas de ALS y permite utilizar los mismos métodos para calcular
el umbral de error y métricas de calidad en ellas, lo que permite una mejor
comparación de los diferentes enfoques de ALS. Este trabajo es fundamental para
este proyecto, ya que se realizó sobre sobre herramienta AxLS, aportando un
enfoque de ML a la herramienta.

La intersección de aplicar técnicas de ML para ALS es un nicho que ha surgido
principalmente en los últimos 5 años. Dentro de esta área se han identificado 3
categorías principales de aplicación de ML a ALS.

La primer categoría identificada es sobre métodos que buscan entrenar modelo
que funcione como ``asistente'' en otros métodos de ALS, evaluando más
rápidamente los efectos que causaría generar cambios en un circuito, usualmente
ayudando a evaluar error rápidamente. El beneficio es que en muchos métodos se
generan cambios a prueba y error o con heurísticas, pero se debe evaluar el
error introducido por cada cambio lo cual puede conllevar simulaciones costosas
computacionalmente, pero un modelo puede estimar el error con una exactitud
aceptable mucho más rápido. Dentro de esta categoría se encuentran los trabajos
de Pasandi et al. \cite{pasandi_approximate_2019}
\cite{pasandi_deep-powerx_2020}, en donde aplica técnicas de aprendizaje
reforzado y aprendizaje profundo (DL por sus siglas en inglés),
respectivamente, para estimar el error generado en un circuito al hacer cambios
locales. En \cite{ye_timing-driven_2024}, Ye et al. no estiman directamente el
error con ML, pero entrenan un agente de aprendizaje reforzado para realizar
cambios locales en un circuito y los estados de este agente son representados
con una red neuronal de grafos con pesos en los caminos (PGNN por sus siglas en
inglés).

La segunda categoría identificada es sobre métodos que se enfocan en técnicas
de ML para realizar exploraciones del espacio de diseño de un circuito.
Específicamente suelen emplear una técnica llamada Búsqueda en Árbol de Monte
Carlo (MCTS por sus siglas en inglés). El trabajo de Rajput et al.
\cite{rajput_improved_2023} plantea la aplicación de MCTS para ALS, modificando
el algoritmo levemente para que pueda explorar nodos más profundos en el árbol
y utilizando la reducción de área como recompensa para evaluar nodos en el
árbol. También se cuenta con \cite{awais_deepapprox_2024}, donde Awais et al.
también aplican MCTS al problema de ALS utilizando ML para la estimación de
error, lo cual lo hace también un ejemplo de la primer categoría de
aplicaciones de ML a ALS mencionada.

La tercer categoría identificada conlleva realizar un entrenamiento supervisado
sobre las entradas y salidas de un circuito, el modelo de ML entrenado es
seguidamente mapeado a un circuito. Ya que los modelos de aprendizaje
supervisado aprenden a generalizar una función, al ser mapeados a un circuito
se obtiene un circuito que aproxima al original.

Uno de los primeros ejemplos en esta categoría es
\cite{boroumand_learning_2021}, donde Boroumand et al. desarrollan un método
para aprender funciones lógicas de ejemplos y sintetizar circuitos basado en el
modelo aprendido, a lo que le llaman síntesis lógica a partir de ejemplos. En
\cite{de_abreu_fast_2021}, De Abreu et al. exploran el uso de árboles de
decisiones (DT por sus siglas en inglés) como alternativa tanto para optimizar
circuitos exactos como para generar versiones aproximadas de los circuitos. En
ambos casos la técnica es exitosa, reduciendo el tiempo de ejecución para
optimizar los circuitos exactos al compararse las alternativas comunes ABC y
Espresso, así como siendo capaz de generar circuitos aproximados de profundidad
y área sumamente reducidos e igual logrando buenos niveles de exactitud.
Miayasaka et al. prueban y comparan varios métodos populares de aprendizaje
supervisado en \cite{miyasaka_logic_2021}, incluyendo redes neuronales, arboles
de decisiones y redes de tablas de búsqueda (LUT por sus siglas en inglés).
Evalúan estos métodos en su capacidad de aproximar circuitos lógicos y
aritméticos, notando que estos modelos comunes no pueden aprender de manera
efectiva ciertos tipos de circuitos aritméticos.

Los resultados del concurso del International Workshop on Logic \& Synthesis
(IWLS) del 2020 \cite{rai_logic_2021} han sido sumamente claves en el avance de
esta área. El concurso consistía en implementar 100 funciones booleanas dados
ejemplos incompletos de sus tablas de verdad, luego las implementaciones fueron
validadas con el set de ejemplos de validación que no fue proporcionado a los
concursantes. Entre las técnicas evaluadas se encuentran DTs, bosques
aleatorios (RF por sus siglas en inglés), redes de LUTs, Espresso, redes
neuronales y programación genética cartesiana. (CGP por sus siglas en inglés)

% Las principales conclusiones del análisis de resultados son que ninguna
% técnica dominó todas las pruebas; la mayoría de los equipos, incluyendo el
% ganador, emplearon un conjunto de técnicas diferentes. Los bosques aleatorios
% y los árboles de decisión fueron muy populares y constituyen un punto de
% referencia sólido para ALS. Sacrificar un poco de exactitud permite una
% reducción significativa en el tamaño del circuito.

Zeng et al. \cite{zeng_sampling-based_2021} explora con mayor profundidad el
uso de DTs para ALS, utilizando una alteración de la técnica llamada árbol de
decisión adaptable, particularmente aplicando variaciones guiadas por una
métrica llamada ``Shapley Additive Explanations'' (SHAP), que busca explicar la
importancia de las características de entrada a un modelo. En la misma línea de
explorar variaciones a la técnica de DTs, Huang y Jiang
\cite{huang_circuit_2023} exploran el uso de grafos de decisión para relajar
las limitaciones estructurales de un árbol, como el crecimiento exponencial a
medida que aumenta la complejidad. Otra variación novedosa de los DTs es la
propuesta de Hu y Cai \cite{hu_optdtals_2024}, donde aplican una técnica a la
que llaman árboles de decisión óptimos, proporcionando una garantía de
optimalidad, lo que les permite un mejor control en el balance entre precisión
y complejidad del circuito.

También se ha profundizado en la técnica de CGP. Berndt et al.
\cite{berndt_cgp-based_2022} proponen recibir una tabla de verdad que
representa el comportamiento deseado y evolucionar circuitos para ajustarse a
ese comportamiento, partiendo de circuitos aleatorios o de un circuito
previamente especificado. Una ventaja de esta técnica es que puede combinarse
con otras, utilizando circuitos generados por otras técnicas como punto de
partida en el proceso evolutivo.

Prats Ramos et al. \cite{prats_ramos_impact_2024} presentan un análisis
comparativo entre las técnicas de DT, CGP y un enfoque mixto de ML que fue
originalmente presentado por el equipo ganador del concurso de IWLS 2020
\cite{rai_logic_2021} y combina el uso de redes neuronales y LUTs.

\section{Planteamiento del problema}

En esta sección se describe el problema central que motiva este proyecto. Se
presenta el contexto en el que surge, su justificación desde una perspectiva
investigativa, y se formula claramente el problema que se busca resolver.

\subsection{Contexto del problema}

Los investigadores del Instituto Tecnológico de Costa Rica enfrentan
limitaciones en el uso de técnicas de ALS basadas en ML. Actualmente, la
herramienta disponible para ellos, AxLS, no cuenta con las capacidades técnicas
necesarias para implementar estas técnicas, lo que restringe sus posibilidades
de experimentación y análisis.

\subsection{Justificación del problema}

La capacidad de utilizar técnicas de ALS basadas en ML permitiría a los
investigadores verificar y comparar resultados obtenidos por otros grupos de
investigación. Además, facilitaría el diseño y desarrollo de técnicas novedosas
de ALS basadas en ML, permitiéndoles contribuir de manera más significativa en
la investigación en ALS.

\subsection{Enunciado del problema}

Los investigadores del Instituto Tecnológico de Costa Rica carecen de una
herramienta que les permita aplicar técnicas de ALS basadas en aprendizaje
automático. La herramienta disponible, AxLS, no posee las capacidades técnicas
necesarias para ello, lo que impide la validación, comparación y desarrollo de
nuevas técnicas en este campo.

\section{Objetivos del proyecto}

\subsection{Objetivo General}

Formular una técnica del estado del arte de aprendizaje automático para ALS en la herramienta AxLS, de manera que permita ser comparada con otras técnicas implementadas dentro de la herramienta así como con otras implementaciones diferentes de las técnicas seleccionadas.

\subsection{Objetivos Específicos}

\begin{enumerate}
  \item Evaluar al menos 3 técnicas de ALS basadas en aprendizaje automático
  determinando cuál es la más apropiada para ser integrada en la herramienta
  AxLS.
  \item Adaptar la herramienta AxLS para que sea posible la implementación de la
  técnica escogida, con la consideración de que permita más fácilmente la
  implementación de futuras técnicas de aprendizaje automático.
  \item Evaluar los resultados de la técnica de aprendizaje automático implementada
  en AxLS con respecto a resultados obtenidos mediante las técnicas ya
  existentes en la herramienta.
\end{enumerate}


\section{Alcances, entregables y limitaciones del proyecto}

El alcance de este proyecto incluye investigar y evaluar la técnicas del estado
del arte, escoger e implementar una en la herramienta AxLS y documentar el
proceso. Los entregables asociados a este alcance, las técnicas y herramientas a emplear así
como sus estrategias de verificación se presentan en la Tabla \ref{tab:entregables}.

Como limitación, la implementación escogida está restringida a las capacidades
actuales de AxLS, incluyendo las métricas y conjunto de pruebas con las que
cuenta actualmente. Este proyecto no plantea agregar más pruebas a la
herramienta.
Otra limitación del proyecto es la falta de acceso a máquinas con capacidades
computacionales de alto rendimiento, por lo que la técnica seleccionada no
puede requerir un entrenamiento demasiado pesado. Debe ser factible su
entrenamiento en poco tiempo en un computador personal.

\begin{table}[htb]
  \caption{Entregables del proyecto.}
  \label{tab:entregables}
  \resizebox{\textwidth}{!}{
    \begin{tabular}{M{0.32\linewidth}M{0.17\linewidth}M{0.37\linewidth}M{0.12\linewidth}}
      \toprule
      Entregable & Técnicas / Herramientas & Estrategias de verificación & Objetivo Específico \\
      \midrule
      Documento de diseño que justifica la escogencia de la técnica de aprendizaje automático a
      implementar en AxLS. Compara todas las técnicas que estén bajo consideración inicialmente
      de aquellas encontradas en trabajos relacionados del estado del arte.
      & Zotero para manejo de referencias.
      & Documento detalla y escoge una técnica en específico, da razones de por qué se escogió
      tomando en cuenta la factibilidad y relevancia en el estado del arte.
      & 1 \\ \\
      Documento de diseño que explica el plan de implementación de la técnica escogida dentro de
      la herramienta AxLS.
      & AxLS, Python, Yosys.
      & Documento detalla todos los cambios de API necesarios en AxLS y el API para utilizar la
      técnica nueva, con ejemplos de uso y un plan de pruebas. Detalla el proceso de crear los
      datos de entrenamiento, entrenamiento del modelo y sintetización del circuito.
      & 2 \\ \\
      Código adaptado de herramienta AxLS con técnica de ML integrada.
      & AxLS, Python, Yosys, git, Github, Pytorch u otra biblioteca de ML.
      & La herramienta AxLS sigue funcionando con todas sus técnicas pasadas y además se puede
      escoger utilizar la técnica de ML y la cual se ejecuta correctamente. La ejecución correcta
      se validará con un plan de pruebas.
      & 2 \\ \\
      Informe final sobre el proyecto realizado y los resultados encontrados al comparar la
      técnica de AA con las ya existentes en la herramienta AxLS
      & LaTeX, IEEE, AxLS
      & El informe describe cómo se llevó a cabo el proyecto, la técnica de ML empleada y
      presenta los resultados de esta técnica comparados con los de las otras técnicas ya
      existentes en la máquina con gráficos.
      & 3 \\
      \bottomrule
    \end{tabular}
  }
\end{table}
