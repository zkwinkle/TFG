\chapter{Conclusiones y recomendaciones}

%%% Lineamientos:
%%% Debe reflejar de forma contundente y ordenada si se lograron los objetivos
%%% propuestos y si se cumplieron los entregables que se habían definido

%%% Lineamientos:
%%% Además, debe mostrar aquellos hallazgos que se consideran relevantes como
%%% experiencia que deja en proyecto desde la perspectiva de la práctica
%%% ingenieril.

% Se evaluaron la mayoría de técnicas de ALS basadas en ML del estado del arte.

Se lograron los objetivos planteados en el proyecto.
% Se escogió DT porque ...
Se evaluaron técnicas de ALS basadas en ML del estado del arte, considerando su
viabilidad práctica y relevancia en el estado del arte.
Basado en esto se escogió la técnica DT por su rapidez de entrenamiento, fácil
mapeo a circuitos y presencia destacada en la literatura.
% Se agregó una clase para el entrenamiento fácil de un DT y mapeo a verilog
Se implementó una clase que facilita el entrenamiento de un DT y su conversión
a un circuito en forma de un módulo de Verilog. Esta clase también sirve como
un ejemplo que sienta las bases de cómo implementar métodos de aprendizaje
supervisado dentro de AxLS a futuro.
% Creación de API simplificada para ejecutar diferentes métodos y CLI para
% experimentación
Además, se creó una API simplificada y una CLI para ejecutar diferentes
métodos, lo que permite una experimentación mucho más ágil y una curva de
aprendizaje menor para utilizar AxLS de manera básica.

% Se comparó el método DT contra métodos de poda disponibles. En métricas salió
% competitivo y en algunos casos dando los mejores resultados.
El método basado en DT fue evaluado frente a tres técnicas de poda disponibles
en AxLS. Los resultados muestran que es competitivo, y en algunos casos logra
los mejores resultados en términos de reducción de área bajo cierto umbral de
error.

% También se evaluó las diferencias en el método DT, se vio que la resíntesis
% puede dar algunas mejoras pequeñas en algunos casos pero no da mucho
% beneficio.
Se exploraron variantes del método DT. La resíntesis aportó pequeñas mejoras en
algunos casos, pero no fue decisiva.
% Se vio que 1 arbol multi-salida genera circuitos más pequeños pero no es
% contundentemente mejor.
Usar un único árbol multi-salida genera circuitos más pequeños y con menor
error que utilizar un árbol diferente para cada bit de salida de un circuito,
pero no de manera contundente. Se observó que utilizar múltiples árboles puede
generar circuitos que a pesar de ser más grandes, o hasta tener peor métricas
de error, capturan mejor los comportamientos complejos de un circuito.

% Quitado por recomendación de lector Pablo García
%
% % En aplicación práctica se vio que puede generar circuitos mejor adaptados
% % usar un árbol diferente por salida.
% Por ejemplo, en un sumador aproximado aplicado a un filtro de desenfoque
% gaussiano, el circuito generado con múltiples árboles tenía peor MRED y
% ocupaba más área que el circuito aproximado con un solo árbol. Sin embargo,
% visualmente logró preservar mejor el brillo y los colores de la imagen
% original.

% En muchos circuitos el método de DT no pudo generar resultados aceptables (de
% área menor a 100% y error menor a 50%)
Se encontró que en varios circuitos el método DT no logró resultados aceptables
bajo las métricas de producir un circuito con área menor al 100\% de la
original con error MRED menor al 50\%. Esto se vio con tres circuitos de
aplicación muy diferente, al ser los circuitos más grandes y complejos.
% En general es difícil generar un circuito aproximado que sí reduzca el área
% con un umbral de error controlado. ESTO ES MUY IMPORTANTE ES EL DOWNSIDE MÁS
% Y MUY CRÍTICO DEL MÉTODO
La limitación más importante identificada en el método de DT es la dificultad
para controlar el error introducido en el circuito generado.
Sin embargo, una ventaja es que es posible ajustar parcialmente el rendimiento
del circuito sobre ciertos rangos de datos, controlando el set de entrenamiento
del DT.
Se recomienda que futuros estudios indaguen más sobre esta propiedad,
explorando la capacidad de manipular los datos de entrenamiento para adaptar el
rendimiento del circuito generado a las necesidades específicas de su
aplicación final.
Esta misma estrategia puede ser relevante para otros métodos de aprendizaje
supervisado.

% (No sé bien en qué parte poner esto) Para el método probprun generar el SAIF
% es un cuello de botella que se mejoro con Rust pero aún podría mejorarse. Si
% se vuelve suficientemente rápido podría mejorar el método probprun por
% permitirle re-evaluar la información de temporización conforme va podando.
Durante la experimentación, también se identificó que la generación de archivos
SAIF para el método \texttt{probprun} es un cuello de botella.
Se logró reducir el tiempo de creación de los archivos SAIF aproximadamente a
un 5\% del tiempo original con una reimplementación en Rust de la utilidad que
los crea, aunque aún existe margen para optimización.
Si este proceso se vuelve suficientemente rápido, comparable a ejecutar una
simulación, podría mejorar la efectividad del método \texttt{probprun} al
permitirle reevaluar información de temporización durante la poda.

%%% Lineamientos:
%%% En lo que respecta a las recomendaciones, se deben incluir los aspectos en
%%% que no se ahondó en el estudio pero que se consideran relevantes y
%%% quedaron como trabajos futuros, así como otras recomendaciones que el
%%% estudiante considere importantes para la continuación y aprovechamiento
%%% de los resultados obtenidos con su proyecto.

% Buscar otros métodos ML que proporcionen un mejor control del error
% introducido.
Se recomienda investigar otras técnicas de ML que ofrezcan mejor control sobre
el error introducido. Para esto, se podría explorar métodos con múltiples
hiperparámetros ajustables, como los MLP.
% También podría ayudar buscar métodos ML cuya estructura se asemeje mejor a la
% de un circuito, en vez de una estructura de árbol. Por ejemplo DG (decision
% graph, poner cita) o CGP (Cartesian Genetic Programming).
También sería valioso explorar modelos cuya estructura se asemeje más a la de
un circuito, como los DG o la CGP.
Los experimentos realizados indican que el método DT no se adapta bien a
circuitos grandes, posiblemente porque su estructura de árbol no refleja la de
estos circuitos.
% Quitado por recomendación de lector Pablo García
%
% Por ejemplo, a mayor cantidad de variables de entrada, el árbol crece
% exponencialmente, y también le resulta difícil aprender operaciones con baja
% información mutua entre entradas y salidas, como el XOR.

% Usar sets de datos más representativos de manera más equitativa para
% diferentes rangos de números. O utilizar ejemplos prácticos y escoger sets de
% datos adecuados.
También se sugiere utilizar sets de datos más representativos y balanceados, o
ejemplos prácticos con conjuntos de datos adecuados para cada caso.
% En general sería mejor utilizar ejemplos prácticos para evaluar los
% resultados y no quedarse con métricas.
En general, es preferible utilizar ejemplos prácticos para evaluar los métodos,
más allá de métricas abstractas.

% También buscar más circuitos de diferentes tipos, más circuitos pequeños para
% que sea más fácil usarlos en pruebas.
Finalmente, sería útil ampliar el set de circuitos de prueba, incluyendo más
circuitos de diferentes tipos y de menor tamaño, para facilitar pruebas rápidas
y exploración.
