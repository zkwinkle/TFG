\chapter*{Resumen}

Este trabajo integra por primera vez técnicas de aprendizaje automático en la
herramienta de síntesis lógica aproximada AxLS para facilitar la generación de
circuitos aproximados. Se identificó y justificó la elección de implementar un
método basado en árboles de decisión por su rapidez de entrenamiento, facil
mapeo a Verilog y relevancia en la literatura. Se extendió AxLS con una clase
para el fácil entrenamiento de árboles de decisiones y generación de módulos de
Verilog aproximados, posibilitando comparaciones directas en métricas de error,
área y tiempo de ejecución. Para evaluar la solución se diseñaron experimentos
automáticos sobre un conjunto de circuitos de referencia, cuantificando la tasa
de error, la reducción de área y la duración del proceso de síntesis.

Los resultados muestran que el método basado en árboles de decisión es
competitivo frente a los métodos de poda existentes en AxLS y puede descubrir
soluciones más eficientes en diversos casos. Sin embargo, presenta menor
control sobre el error introducido y, dependiendo del circutio, no siempre
logra generar aproximaciones más compactas que el diseño original.

Esta implementación no solo amplía las capacidades de AxLS, sino que sienta las
bases para incorporar otras técnicas de ML y fomentar la adopción de técnicas
ML dentro de flujos de ALS.

\paragraph{Palabras clave}

Diseño electrónico automatizado, síntesis lógica aproximada, aprendizaje
automático, árboles de decisión.


\chapter*{Abstract}

This work integrates machine learning techniques into the AxLS approximate
logic synthesis tool to enable the generation of approximate circuits. A
decision tree based method was selected and justified due to its fast training,
straightforward mapping to Verilog, and strong presence in the literature. AxLS
was extended with a class for easy training of decision trees and automated
generation of approximate Verilog modules, allowing direct comparisons in
error, area, and synthesis time metrics. To evaluate the approach, automated
experiments were conducted on a set of benchmark circuits, measuring error
rates, area reduction, and synthesis duration.

Results show that decision trees are competitive with existing pruning methods
in AxLS and can yield more efficient solutions in several cases. However, they
provide less control over the introduced error and may not always produce
smaller approximations than the original design, depending on the circuit.

This work extends the AxLS framework and opens the door to incorporating a
wider range of ML based approaches into ALS workflows.

\paragraph{Key words}

Electronic Design Automation, Approximate Logic Synthesis, Machine Learning, Decision Trees.
