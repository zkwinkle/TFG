\chapter{Descripción del trabajo realizado}

Este capítulo consta de dos secciones principales. La primera presenta el
proceso seguido en función del marco metodológico: actividades realizadas,
retos encontrados, decisiones de diseño tomadas. La segunda sección presenta el
análisis de los resultados obtenidos por el proyecto, enfocado en comparar el
método de ML integrado en AxLS con otros métodos previamente existentes.

\section{Descripción del proceso de solución}

El proceso para llegar a la solución siguió la estructura planteada en el marco
metodológico, que la Figura \ref{fig:diagrama_metodológico} resume visualmente.

\subsection{Experimentación con AxLS}

El objetivo de esta actividad fue familiarizar al autor con la herramienta
AxLS, sus módulos existentes y formar una mejor idea de lo que tomaría el
proyecto. Se decidió intentar crear una prueba de concepto de lo que sería la
verdadera implementación de ML dentro de la herramienta, como ejercicio de
familiarización.
En el lapso de una semana se intentaría implementar un DT. Se
escogió utilizar un DT porque es una técnica fácil de utilizar correctamente,
cuya implementación ya existe en la biblioteca \emph{scikit-learn} y que se
entrena rápidamente.
La prueba de concepto fue exitosa, se logró realizar una síntesis aproximada de
uno de los circuitos de benchmark y se identificó cómo realizar los pasos
necesarios para la implementación de la técnica:

\begin{itemize}
    \item Sintetización del circuito original.
    \item Simulación del circuito original para obtener los datos de entrenamiento.
    \item Lectura de los datos de entrada y salida del circuito original para
      entrenar el árbol de decisiones.
    \item Entrenamiento del árbol de decisiones con implementación de
      \emph{scikit-learn}.
    \item Mapeo del árbol de decisiones a un circuito Verilog a través de
      expansión de Boole.
    \item Sintetización y simulación del circuito aproximado.
    \item Evaluación de técnicas de error.
\end{itemize}

Cabe notar que la técnica no estaba realmente integrada dentro de la
herramienta, ya que fue una implementación ad hoc. Además, presentaba varios
errores. Se decidió no dedicar más tiempo a intentar corregirlos, porque no
había certeza de que esa fuera la técnica final que se escogería para agregar
propiamente a AxLS.

\subsection{Escogencia de método ML}

Para escoger el método de ML, se siguieron los criterios planteados en el marco
metodológico, en la sección \ref{sec:seleccion_ml}. Después de una revisión
bibliográfica exhaustiva sobre técnicas de ML aplicadas a ALS, se realizó un
análisis explorando las diferentes técnicas utilizadas en el estado del arte.
Finalmente, se justificó cuál era la más apropiada para este proyecto.

\subsubsection{Escogencia de categoría}

Como se discute en la sección \ref{sec:trabajos_similares}, en la bibliografía
se identificaron 3 categorías principales de aplicación de métodos de ML en
ALS:

\begin{enumerate}
    \item Técnicas que entrenan modelos para asistir en otros métodos de ALS,
      acelerando la evaluación de errores al simular cambios en el circuito.
    \item Técnicas de ML aplicadas a la exploración del espacio de diseño de
      circuitos.
    \item Enfoques supervisados donde se entrena un modelo con entradas/salidas
      de un circuito, y luego se mapea a hardware.
\end{enumerate}

Para el método a implementar en la herramienta AxLS se escogió la categoría 3
debido a las siguientes consideraciones:

\begin{itemize}
    \item Se descartó completamente la categoría 2 por el criterio de
      viabilidad, ya que las exploraciones o entrenamientos realizados duran
      horas y son realizados en estaciones de trabajo considerablemente más
      potentes que lo que se tiene acceso para este proyecto.
    \item Luego se evaluaron más detenidamente los criterios de escogencia entre
      las categorías 1 y 3:
    \begin{itemize}
        \item Viabilidad: Por lo general los métodos de la categoría 3 tienen
          un entrenamiento menos extenso y más fáciles de realizar.
        \item Estado del arte: Hay considerablemente más trabajos de la
          categoría 3.
        \item Resultados: En ambas categorías hay resultados competitivos,
          ninguna es considerablemente más exitosa que la otra.
    \end{itemize}
\end{itemize}

\subsubsection{Escogencia de método}

Se evaluaron los métodos disponibles dentro de la categoría 3, y sus
consideraciones se discuten a continuación. En la sección
\ref{sec:trabajos_similares} se adentra más en los métodos de las otras
categorías.

\paragraph{Árbol de decisiones}

\begin{itemize}
    \item Es el método más estudiado en la literatura, 7 de las 10 referencias
      utilizadas lo estudian \cite{de_abreu_fast_2021, miyasaka_logic_2021,
      rai_logic_2021, zeng_sampling-based_2021, huang_circuit_2023,
      hu_optdtals_2024, prats_ramos_impact_2024}.
    \item Es muy viable debido a su fácil implementación, disponibilidad en
      bibliotecas de alto grado como \emph{scikit-learn} y bajo tiempo de
      entrenamiento. En efecto, ya se había realizado una implementación ad hoc
      como una prueba de concepto a inicios del proyecto.
    \item Tiene resultados competitivos. No siempre los mejores, pero por lo
      general la literatura lo menciona como una de las técnicas más efectivas
      o la base que están intentando superar con un método más avanzado.
\end{itemize}

Se consideró la posibilidad de implementar alguna de las versiones modificadas
que se han estudiado, como en \cite{hu_optdtals_2024} o
\cite{zeng_sampling-based_2021}, pero se decidió sería mejor implementar el
método más general para tener una base de comparación y dejar cualquier mejora
para un trabajo futuro.

\paragraph{Bosque aleatorio}

\begin{itemize}
    \item Fue explorada en 2 de las referencias estudiadas, las cuales son del
      2021 \cite{miyasaka_logic_2021}, \cite{rai_logic_2021}.
    \item Son fáciles de implementar, ya que están compuestos por DT.
    \item Sí ha obtenido mejores resultados que los DT en algunas de las tareas
      evaluadas dentro de las referencias, particularmente a la hora de
      generalizar circuitos lógicos y para algunos circuitos aritméticos en
      específico.
\end{itemize}

\paragraph{Programación genética cartesiana}

\begin{itemize}
    \item Esta técnica usa algoritmos evolutivos para generar circuitos,
      representando sus elementos como un \emph{genotipo} fácilmente mapeable
      al circuito final.
    \item Ha sido estudiada en 2 de las referencias utilizadas, las cuales son
      bastante modernas con la última siendo del 2024
      \cite{berndt_cgp-based_2022}, \cite{prats_ramos_impact_2024}.
    \item Esta técnica obtiene resultados muy exitosos en términos del
      intercambio entre área de circuito y error introducido.
    \item Se descartó por su alto costo computacional: incluso con un clúster
      mucho más potente que el equipo disponible, en
      \cite{berndt_cgp-based_2022} las ejecuciones tomaron hasta un día
      completo.
\end{itemize}

\paragraph{Perceptrón multicapa}

\begin{itemize}
    \item Esta técnica es explorada en 4 de las referencias utilizadas, con 3
      de ellas siendo del 2021 y la última del 2024
      \cite{boroumand_learning_2021}, \cite{miyasaka_logic_2021},
      \cite{rai_logic_2021}, \cite{prats_ramos_impact_2024}
    \item Su viabilidad es limitada por la dificultad de traducirlas a
      circuitos. Suelen requerir módulos complejos como sumadores y
      multiplicadores, lo que aumenta mucho el tamaño. Reducir esto implica
      técnicas como podar conexiones poco importantes o convertir cada nodo del
      MLP en una LUT mediante cuantización.
    \item Tienen buenos resultados en sus estudios, particularmente con
      circuitos aritméticos de adición y multiplicación y aprendiendo la
      operación lógica de XOR, tareas en las que los otros métodos suelen
      fallar.
\end{itemize}

\paragraph{Método escogido}

Tomando en cuenta las consideraciones dadas para cada método se decidió
implementar DT, principalmente por su alta viabilidad y relevancia dentro de la
literatura.

Se considera que además es una buena base para trabajo futuro en el que se implemente alguna variación de la técnica como RF, o las presentadas en \cite{hu_optdtals_2024}, \cite{zeng_sampling-based_2021} o \cite{huang_circuit_2023}.


\section{Implementación de técnica de ML}

Debido a la implementación previa de la técnica de DT como prueba de concepto,
la implementación real fue más fácil. Aun así, fue necesario integrarla
correctamente en AxLS, corregir errores y generalizarla para que funcionara con
cualquier circuito, no solo uno específico.

Se creó una clase llamada \texttt{DecisionTreeCircuit} que abstrae sobre la
implementación de DT de \emph{scikit-learn} para dar las utilidades necesarias
para realizar ALS. Esta solo expone los siguientes métodos:

\begin{itemize}
  \item \texttt{train(X, y)}: Acepta vectores con los datos obtenidos de la
    simulación del circuito original que utiliza para entrenar el árbol de
    decisiones interno.
  \item \texttt{to\_verilog\_file(topmodule, filename)}: Después de entrenar el
    árbol de decisiones, este método mapea el árbol entrenado a un circuito de
    Verilog y lo escribe en un archivo de texto especificado por el usuario. El
    parámetro de \texttt{topmodule} permite al usuario controlar el nombre del
    módulo de Verilog para el circuito.
\end{itemize}

Para que el árbol se adapte al circuito que se está aproximando y el módulo de
Verilog se genere correctamente, se deben suplir listas con las entradas y
salidas del circuito.
También se agregó el parámetro \texttt{one\_tree\_per\_output}, que permite
elegir entre un solo árbol multi-salida o uno por cada salida del circuito.
Esto se incluyó tras notar en la prueba de concepto que ambas opciones eran
posibles y afectaban el resultado.

La clase, sus propiedades y métodos se muestran en formato UML en la Figura
\ref{fig:UML}.

\begin{figure}[htb]
  \centering
  % El inkscapelatex=false evita que LaTeX trate de renderizar el texto,
  % necesario porque estaba tirando error por los underscores '_' en el svg.
  \includesvg[inkscapelatex=false, width=0.3\linewidth]{./imágenes/DecisionTreeCircuit_UML.svg}
  \caption{Representación de UML de \texttt{DecisionTreeCircuit}.}
  \label{fig:UML}
\end{figure}

El flujo completo de utilizar esta clase con las utilidades de AxLS conlleva los siguientes pasos, los cuales se representan visualmente en la Figura \ref{fig:flow}.

\begin{enumerate}
  \item Sintetizar circuito original con la clase \texttt{Circuit} ya existente dentro de AxLS.
  \item Simular el circuito original con un set de datos de entrada, generando las salidas correspondientes.
  \item Leer los archivos del set de datos de entrada y sus salidas correspondientes y entrenar el árbol de decisiones con estos.
  \item Mapear el árbol de decisiones a un circuito aproximado de Verilog.
  \item Sintetizar el circuito aproximado de Verilog con la clase \texttt{Circuit}. Con este objeto se pueden realizar simulaciones para obtener las métricas de área y error del circuito aproximado.
\end{enumerate}

\begin{figure}[htb]
  \centering
  % El inkscapelatex=false evita que LaTeX trate de renderizar el texto,
  % necesario porque estaba tirando error por los underscores '_' en el svg.
  \includesvg[width=0.6\linewidth]{./imágenes/decision_tree_method.svg}
  \caption{Representación visual de cómo utilizar \texttt{DecisionTreeCircuit} dentro de AxLS.}
  \label{fig:flow}
\end{figure}

Para validar la implementación se hicieron pruebas con un circuito sumador de 9
bits de entrada (dos entradas de 4 bits y 1 bit de acarreo). Las pruebas
examinaron los productos intermedios generados y el rendimiento del circuito
final para verificar su correcta implementación. Varias de las pruebas se
hicieron utilizando un DT sin profundidad máxima, que para este pequeño
circuito es capaz de memorizar todos los pares de entradas y salidas, generando
resultados exactos.

\begin{itemize}
  \item Se verificó que las entradas y salidas del circuito de Verilog generado
    calzaban correctamente a través de inspección manual y revisando que las
    salidas al simular el circuito coinciden con las salidas al manipular el DT
    en software directamente.
  \item Utilizando un DT sin profundidad máxima, se verificó en software que
    las salidas correspondían con lo esperado dependiendo de las entradas. Por
    ejemplo si las entradas son ${3, 4, 1}$, la salida sería $8$ en binario.
  \item Utilizando un DT sin profundidad máxima, el circuito aproximado debería
    tener error de $0\%$ y se esperaría que tenga un área mayor al circuito
    original, ya que memorizar los pares de entradas y salidas es menos
    eficiente que programar la estructura de un sumador.
  \item Al reducir la profundidad máxima del DT, el área del circuito generado
    debía disminuir y el error aumentar.
\end{itemize}


\subsection{Recolección de resultados.}

 TODO

 1. Se definieron las tablas de resultados que se deseaban generar.

 2. Se implementaron utilidades para fácilmente generar sets de datos para cada circuito y para ejecutar cualquier método de la herramienta.

 3. Se creó una utilidad para pasar de vcd a saif en Rust para realizarlo más rápido.

   - Incluir banderas de configuración.

   - Incluir facilidad para separar el set de datos en prueba y verificación.

 4. Después de exprimentar con todos los métodos de la herramienta, se
 decidieron excluir ccarving y significance del estudio porque requieren mucha
 configuración y prueba y error para obtener los mejores resultados y se decidió
 que esto se sale del alcance.

 5. Se decidió limitar los sets de datos dependiendo de cuánto duren en simularse.
   - Primero dar introducción de los circuitos benchmarks a utilizar y el tipo de nombres con el que se refieren.

 6. Se realizaron simulaciones cortas de todos los circuitos para obtener información del tiempo que duran en simularse.

 7. Se ajustaron los sets de datos para ser de 2 segundos, debido a los métodos de poda.

 8. Bajo esta métrica, los circuitos grandes tendrían sets de datos muy pequeños y no representativos, entonces se excluyeron.

 9. También se excluyeron el fir y el invk2j debido a implementaciones erróneas.

 10. También se excluyeron algunos por durar demasiado en la poda, culpa de usar métrica mred con valores muy grandes.

\section{Análisis de los resultados obtenidos}

