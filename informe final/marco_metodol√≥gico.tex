\chapter{Marco metodológico}

Este capítulo detalla la metodología seguida para el desarrollo del
proyecto. Busca que el lector comprenda cómo se estructuró el proceso
de diseño, implementación y validación de la solución propuesta.

El proyecto contó con las siguientes etapas principales:

\begin{enumerate}
  \item Experimentación y familiarización con AxLS.
  \item Selección del método ML a implementar dentro de AxLS.
  \item Implementación de técnica de ML dentro de AxLS.
  \item Diseño de experimentos y recolección de resultados.
\end{enumerate}


\section{Estrategia para la solución}

En alineación con los objetivos específicos, este proyecto cumplió con las
siguientes tareas clave:

\begin{itemize}
  \item Escoger una técnica de ML para ALS lo más del estado del arte posible.
  \item Implementar esta técnica como un método de ALS disponible dentro de la
    herramienta de fuente abierta AxLS.
  \item Utilizar AxLS para comparar los resultados obtenidos por la nueva
    técnica ML con los de otras técnicas ya previamente disponibles dentro de
    la herramienta.
\end{itemize}


\begin{figure}[htb]
  \begin{center}
    \includesvg[width=0.95\textwidth]{./imágenes/diagrama_marco_metodológico.svg}
  \end{center}
  \caption{Diagrama de la secuencia de pasos seguidos y los productos de cada etapa.}
  \label{fig:diagrama_metodológico}
\end{figure}

Cada tarea involucró tanto actividades de diseño como de ejecución, incluyendo
revisión bibliográfica e implementación de código.

Adicionalmente, se designó un tiempo en las semanas iniciales para
experimentación y familiarización con AxLS, con el único objetivo de que el
autor se familiarizara con los flujos de la herramienta y obtuviera una mejor
claridad en los pasos que llegaran a ser necesarios en la implementación e
integración de una técnica ML.
Este fue un periodo puramente exploratorio por lo que no se definió ningún
entregable o actividades claras como tal.


La Figura \ref{fig:diagrama_metodológico} contiene un diagrama que muestra las
diferentes etapas, sus actividades y productos principales.
En el resto de esta sección se entra en más detalles sobre cada una de las tareas clave, las
actividades realizadas y productos generados.
También, la Tabla \ref{tab:actividades} muestra un resumen de las principales actividades asociadas a
cada objetivo específico y los entregables producidos.

\subsubsection{Selección de técnica ML}
\label{sec:seleccion_ml}

Para la selección de la técnica ML, primero se realizó un estudio exhaustivo
de la literatura disponible sobre aplicación de métodos ML en ALS. Estas
referencias son las mencionadas en la sección \ref{sec:trabajos_similares}.
Para este estudio bibliográfico se empleó la herramienta Zotero para un manejo
más fácil de las referencias utilizadas.

Seguidamente, se definieron los criterios para la selección de la técnica ML
más apropiada a implementar en AxLS. Basado en estos criterios de selección, se
elaboró un documento de diseño evaluando los métodos identificados en el
estudio bibliográfico y concluyendo cuál era el más apropiado.

A continuación se detallan los criterios definidos para la selección del método.

\paragraph{Viabilidad}
\begin{itemize}
  \item No se cuentan con computadores de alto rendimiento para llevar a cabo
    entrenamientos extensivos.
  \item AxLS en sí no tiene restricciones que impidan implementar métodos que
    requieran mucho tiempo y procesamiento. Sin embargo, al ser de fuente abierta, podría tener una amplia variedad de usuarios, por lo que sería
    favorable que cualquiera pueda ejecutar la herramienta en su totalidad.
    Que cualquier usuario pueda probar la mayor variedad de métodos posibles
    facilitaría una mayor adopción, lo que a su vez podría resultar en más
    contribuciones y mayor utilidad.
  \item Si es posible ejecutar las pruebas rápidamente, eso es altamente
    preferible para facilitar la verificación del funcionamiento del método
    y su ejecución por cualquier usuario.
    \begin{itemize}
      \item Se definió arbitrariamente, como un valor de tiempo bajo, que la
        ejecución del método seleccionado preferiblemente dure menos de 5
        minutos en un computador personal.
    \end{itemize}

  \item También se consideró la viabilidad de la implementación, ya que este
    proyecto cuenta con poco tiempo para desarrollarse.
\end{itemize}

\paragraph{Estado del arte}
\begin{itemize}
  \item Es preferible implementar un método lo más moderno posible para
    asegurar una mayor relevancia en el estado del arte.

  \item Es preferible implementar un método con amplia base de estudios
    previos, como heurística de su relevancia y para contar con más puntos de
    comparación en la literatura.
\end{itemize}

\paragraph{Resultados}
\begin{itemize}
  \item Se dará menor importancia a los resultados obtenidos por la técnica
    en la literatura, ya que este trabajo tiene un rol más fundacional:
    implementar técnicas de ML en AxLS, y no necesariamente busca obtener los
    mejores resultados posibles. Sin embargo, en caso de similitud entre
    métodos en los demás criterios, se podrá considerar cuál ha obtenido
    mejores resultados en otros estudios.
\end{itemize}

\subsection{Implementación de técnica ML}

Para la implementación de la técnica ML dentro de AxLS se redactó un documento
de diseño detallando la API para utilizar esta técnica, ejemplos de uso y
diagramas visuales para ayudar la comprensión del flujo para utilizarla.

El flujo debe de tomar en cuenta los siguientes pasos:

\begin{itemize}
  \item Obtención de los datos de entrenamiento.
  \item Entrenamiento del modelo ML.
  \item Utilización del modelo ML dentro de AxLS para obtener circuitos
    aproximados.
    \begin{itemize}
      \item Dependiendo de la técnica seleccionada, este paso puede implicar
        mapear el modelo ML directamente a un circuito. En ese caso, se genera un
        archivo Verilog que luego puede ser procesado por AxLS. Alternativamente,
        el modelo puede integrarse directamente en AxLS como parte de un flujo
        más complejo, por ejemplo, para estimación de error o exploración del
        espacio de diseño.
    \end{itemize}
\end{itemize}

Para verificar la correcta operación de la implementación, se deberá diseñar
una prueba enfocada en validar dos aspectos: primero, que el modelo ML logra
entrenarse y modelar el comportamiento deseado; segundo, que su integración en
AxLS permite generar circuitos aproximados funcionales y comparables en
métricas de área y error introducido frente a otros métodos existentes. Esta
prueba será de escala reducida y con fines de verificación interna, distinta de
los experimentos usados para el análisis cuantitativo presentado en este
informe. Su diseño dependerá de la técnica ML seleccionada.

Para realizar esta implementación y su validación se utilizará el lenguaje
Python, ya que es el lenguaje en el que está escrito la herramienta AxLS.

\subsection{Diseño de experimentos y recolección de resultados}
\label{sec:metodologia_resultados}

\begin{figure}[htb]
  \begin{center}
    \includesvg[width=0.8\textwidth]{./imágenes/Diagrama runner.svg}
  \end{center}
  \caption{Componentes creados para la ejecución de métodos de AxLS. Los bloques azules corresponden a elementos que son parte de AxLS, el script de Python para recolección de resultados sería externo a la herramienta.}
  \label{fig:runner}
\end{figure}

Como parte de esta tarea se definieron los resultados necesarios para comparar
la técnica de ML implementada con los métodos ya disponibles en AxLS.

Aunque los resultados específicos dependerían de la técnica seleccionada, por
ejemplo en caso de requerir variaciones en hiperparámetros, se estableció
desde el inicio que las métricas clave serían: error introducido al circuito,
reducción de área, y tiempo de ejecución.

AxLS no cuenta con una interfaz de ejecución automatizada, sino que funciona
como una biblioteca que debe integrarse manualmente mediante código en Python.
Esto representa un reto al momento de ejecutar múltiples pruebas con distintos
métodos, parámetros y benchmarks.

Por ello, se decidió diseñar una API interna que permitiera configurar y
ejecutar cada técnica de forma simplificada. Esta API fue pensada para
facilitar su exposición a través de una interfaz de línea de comandos (CLI), lo
que permite una experimentación más ágil sin necesidad de escribir código
adicional. Además de facilitar los experimentos, esta interfaz puede fomentar
la adopción de AxLS al reducir la curva de aprendizaje para nuevos usuarios.

Con esta infraestructura, fue posible generar todos los resultados de forma
automatizada mediante un script sencillo en Python.
Además, la CLI resultó fundamental para la ejecución de pruebas de verificación
y la exploración ágil de distintos escenarios.

La Figura \ref{fig:runner} muestra cómo se relacionan los distintos componentes
desarrollados.

Para verificar el funcionamiento correcto de la API y la CLI, se implementaron
scripts de validación que utilizan directamente la biblioteca de AxLS y se
compararon sus resultados con los generados por la nueva interfaz. Dado que
algunos métodos de AxLS no son determinísticos, se consideró válido que las
ejecuciones tengan diferencias menores, tanto en tiempo como en resultados. Se
aceptó un margen de variación del 5\% en métricas de área y error.

\begin{table}[htb]
  \caption{Actividades principales del proyecto. La Tabla \ref{tab:entregables} resume las herramientas utilizadas para cada entregable y sus estrategias de verificación.}

  \label{tab:actividades}
  \resizebox{\textwidth}{!}{

    \begin{tabular}{M{0.32\linewidth}M{0.6\linewidth}M{0.12\linewidth}}
      \toprule
      Entregable & \centering Principales actividades & Objetivo Específico \\
      \midrule
      Documento de diseño que justifica la selección de la técnica ML a implementar en AxLS.
      & \bulletspace Revisión bibliográfica.
        \newline \bulletspace Definición de criterios de selección.
        \newline \bulletspace Selección justificada de técnica ML.
      & 1 \\ \\
      Documento de diseño que explica el plan de implementación de la técnica escogida dentro de la herramienta AxLS.
      & \bulletspace Diseño de API para técnica ML.
      & 2 \\ \\
      Código de herramienta AxLS con técnica de ML integrada.
      & \bulletspace Implementación de técnica ML en AxLS.
        \newline \bulletspace Verificación de método ML.
      & 2 \\ \\
        Código adaptado de herramienta AxLS con adaptaciones para fácil ejecución.
      & \bulletspace Implementación de API simplificada.
        \newline \bulletspace Exponer API simplificada con una CLI.
        \newline \bulletspace Verificación de API simplificada y CLI.
      & 3 \\ \\
      Informe final sobre el proyecto realizado y los resultados encontrados al comparar la
      técnica de ML con las ya existentes en la herramienta AxLS
      & \bulletspace Creación de script para generar resultados.
        \newline \bulletspace Presentación de datos con elementos gráficos.
        \newline \bulletspace Análisis de resultados.
      & 3 \\
      \bottomrule
    \end{tabular}
  }
\end{table}
